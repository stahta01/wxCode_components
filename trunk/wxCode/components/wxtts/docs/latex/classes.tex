\chapter{Alphabetical class reference}\label{classref}

\setheader{{\it CHAPTER \thechapter}}{}{}{}{}{{\it CHAPTER \thechapter}}%
\setfooter{\thepage}{}{}{}{}{\thepage}%

\section{\class{wxFreePhone}}\label{wxfreephone}

The wxFreePhone class allows generation of Mbrola-compatible phoneme
strings or files from textual input.

Note that in this class reference, member functions involving strings are documented
with several forms, since CString, wxString or char* are used depending on how the
library is compiled (MFC, wxWindows or plain WIN32 or Unix). These functions are mutually exclusive
per build.

\wxheading{Derived from}

Nothing

\wxheading{See also}

\overview{wxTTS overview}{ttsoverview}

\latexignore{\rtfignore{\wxheading{Members}}}

\membersection{wxFreePhone::wxFreePhone}

\func{}{wxFreePhone}{\param{const CString\&}{ dictionaryPath = ""}}

\func{}{wxFreePhone}{\param{const wxString\&}{ dictionaryPath = ""}}

\func{}{wxFreePhone}{\param{const char*}{ dictionaryPath = ""}}

Constructor. {\it dictionaryPath} is an optional path and filename (but with no extension)
indicating where the word-to-phoneme dictionary is stored. For example:

\begin{verbatim}
  wxFreePhone freePhone("d:\\wxtts\\dictionary\\english");
\end{verbatim}

allows wxFreePhone to find the files {\tt english.dct}, {\tt english.key} and {\tt english.idx} which are
all required to produce high-quality speech.

Note that before performing phoneme generation, you must call \helpref{wxFreePhone::Init}{wxfreephoneinit} to
actually load the dictionary.

\membersection{wxFreePhone::CleanUp}\label{wxfreephonecleanup}

\func{Bool}{CleanUp}{\void}

Closes the dictionary and frees memory. Called automatically from the wxFreePhone destructor.

\membersection{wxFreePhone::Init}\label{wxfreephoneinit}

\func{Bool}{Init}{\void}

Loads the dictionary which was specified in the constructor, if any. It is necessary
to call this before any phoneme generation is performed.

\membersection{wxFreePhone::TextToPhoneme}\label{wxfreephonetexttophoneme}

\func{Bool}{TextToPhoneme}{\param{const CString\&}{ text}, \param{CString\&}{ phonemes}}

\func{Bool}{TextToPhoneme}{\param{const wxString\&}{ text}, \param{wxString\&}{ phonemes}}

\func{Bool}{TextToPhoneme}{\param{const char*}{ text}, \param{char**}{ phonemes}}

Converts a text string to a phoneme string, in Mbrola format.

The result is returned in the second argument. In the case of a pure Windows build,
new character arrary memory is allocated and returned, and should be deleted (using the array
deletion operator) by the calling application, if the function returned TRUE.

\membersection{wxFreePhone::TextToPhonemeFile}\label{wxfreephonetexttophonemefile}

\func{Bool}{TextToPhonemeFile}{\param{const CString\&}{ text}, \param{const CString\&}{ phonemeFile}}

\func{Bool}{TextToPhonemeFile}{\param{const wxString\&}{ text}, \param{const wxString\&}{ phonemeFile}}

\func{Bool}{TextToPhonemeFile}{\param{const char*}{ text}, \param{const char*}{ phonemeFile}}

Converts a text string to a file of phonemes, in Mbrola format.


\section{\class{wxMbrola}}\label{wxmbrola}

The wxMbrola class controls Mbrola, a high-quality speech synthesizer.
It takes strings or files of phonemes, which are supplied by wxFreePhone.

Note that in this class reference, member functions involving strings are documented
with several forms, since CString, wxString or char* are used depending on how the
library is compiled (MFC, wxWindows or neither). These functions are mutually exclusive
per build.

\wxheading{Derived from}

Nothing

\wxheading{See also}

\overview{wxTTS overview}{ttsoverview}

\latexignore{\rtfignore{\wxheading{Members}}}

\membersection{wxMbrola::wxMbrola}

\func{}{wxMbrola}{\void}

Constructor.

Note that before performing phoneme generation, you must call \helpref{wxMbrola::Init}{wxmbrolainit} to
initialise the speech synthesizer.

\membersection{wxMbrola::Init}\label{wxmbrolainit}

\func{Bool}{Init}{\void}

Initializes the speech synthesizer.

\membersection{wxMbrola::GetCurrentDatabase}\label{wxmbrolagetcurrentdatabase}

\constfunc{Bool}{GetCurrentDatabase}{\param{CString\&}{ db}}

\constfunc{Bool}{GetCurrentDatabase}{\param{wxString\&}{ db}}

\constfunc{Bool}{GetCurrentDatabase}{\param{char*}{ db}}

Gets the current diphone database used by Mbrola. The name {\it db} is a short label such as "en1".

\membersection{wxMbrola::GetDatabaseAllInfo}\label{wxmbrolagetdatabaseallinfo}

\constfunc{CString}{GetDatabaseAllInfo}{\void}

\constfunc{wxString}{GetDatabaseAllInfo}{\void}

\constfunc{char*}{GetDatabaseAllInfo}{\void}

Returns information about the current database. The third form returns a pointer to a newly allocated
character array, which should be deleted by the application.

\membersection{wxMbrola::GetDatabaseId}\label{wxmbrolagetdatabaseid}

\constfunc{Bool}{GetDatabaseId}{\param{int}{ n}, \param{CString\&}{ id}}

\constfunc{Bool}{GetDatabaseId}{\param{int}{ n}, \param{wxString\&}{ id}}

\constfunc{Bool}{GetDatabaseId}{\param{int}{ n}, \param{char*}{ id}}

Gets the nth diphone database name known to wxMbrola. Use \helpref{wxMbrola::GetDatabaseCount}{wxmbrolagetdatabasecount} to
find out how many diphone databases there are. Note that these are Mbrola-specific diphone databases, nothing to do with
the wxFreePhone dictionary.

The char* form of GetDatabaseId expects a character buffer to be passed,
into which the database id will be copied.

\membersection{wxMbrola::GetDatabaseCount}\label{wxmbrolagetdatabasecount}

\constfunc{int}{GetDatabaseCount}{\param{int}{ n}}

Returns the number of phoneme databases known to wxMbrola.

See also \helpref{wxMbrola::GetDatabaseId}{wxmbrolagetdatabaseid}.

\membersection{wxMbrola::GetDefaultDatabase}\label{wxmbrolagetdefaultdatabase}

\constfunc{CString}{GetDefaultDatabase}{\void}

\constfunc{wxString}{GetDefaultDatabase}{\void}

\constfunc{char*}{GetDefaultDatabase}{\void}

Returns the default database. This is not the current database used for synthesis, but the default which
is specified by an optional mbrola.ini file, as a hint.

\membersection{wxMbrola::GetDefaultFreq}\label{wxmbrolagetdefaultfreq}

\constfunc{long}{GetDefaultFreq}{\void}

Returns the default frequency for the current voice.

\membersection{wxMbrola::GetDefaultPath}\label{wxmbrolagetdefaultpath}

\constfunc{CString}{GetDefaultPath}{\void}

\constfunc{wxString}{GetDefaultPath}{\void}

\constfunc{char*}{GetDefaultPath}{\void}

Returns the default path, as specified in mbrola.ini. Not currently used.

\membersection{wxMbrola::GetDuration}\label{wxmbrolagetduration}

\constfunc{float}{GetDuration}{\void}

Gets the current duration of the voice, between around 0 and 10 (a value of 0.2 is reasonable).

\membersection{wxMbrola::GetFreq}\label{wxmbrolagetfreq}

\constfunc{float}{GetFreq}{\void}

Sets the current frequency of the voice, between around 8000 and 32000. A value of 16000 would be reasonable.

\membersection{wxMbrola::GetLastError}\label{wxmbrolagetlasterror}

\constfunc{CString}{GetLastError}{\void}

\constfunc{wxString}{GetLastError}{\void}

\constfunc{char*}{GetLastError}{\void}

Returns a string describing the last error. In the third form, returns a pointer to
a static character array, so do not deallocate it.

\membersection{wxMbrola::GetMaxPitch}\label{wxmbrolagetmaxpitch}

\constfunc{float}{GetMaxPitch}{\void}

Gets the maximum pitch that an application allows the user to have. This value doesn't
affect Mbrola directly, it is up to the application to make use of it. The maximum
pitch is read from the mbrola.ini file if it exists. A reasonable maximum pitch value might be 8.0.

\membersection{wxMbrola::GetMinPitch}\label{wxmbrolagetminpitch}

\constfunc{float}{GetMinPitch}{\void}

Gets the minimum pitch that an application allows the user to have. This value doesn't
affect Mbrola directly, it is up to the application to make use of it. The minimum
pitch is read from the mbrola.ini file if it exists. A reasonable minimum pitch value might be 0.2.

\membersection{wxMbrola::GetMaxDuration}\label{wxmbrolagetmaxduration}

\constfunc{float}{GetMaxDuration}{\void}

Gets the maximum duration that an application allows the user to have. This value doesn't
affect Mbrola directly, it is up to the application to make use of it. The maximum
duration is read from the mbrola.ini file if it exists. A reasonable maximum duration value might be 8.0.

\membersection{wxMbrola::GetMinDuration}\label{wxmbrolagetminduration}

\constfunc{float}{GetMinDuration}{\void}

Gets the minimum duration that an application allows the user to have. This value doesn't
affect Mbrola directly, it is up to the application to make use of it. The minimum
duration is read from the mbrola.ini file if it exists. A reasonable minimum duration value might be 0.2.

\membersection{wxMbrola::GetMaxFreq}\label{wxmbrolagetmaxfreq}

\constfunc{long}{GetMaxFreq}{\void}

Gets the maximum freq that an application allows the user to have. This value doesn't
affect Mbrola directly, it is up to the application to make use of it. The maximum
freq is read from the mbrola.ini file if it exists. A reasonable maximum freq value might be 32000.

\membersection{wxMbrola::GetMinFreq}\label{wxmbrolagetminfreq}

\constfunc{long}{GetMinFreq}{\void}

Gets the minimum freq that an application allows the user to have. This value doesn't
affect Mbrola directly, it is up to the application to make use of it. The minimum
freq is read from the mbrola.ini file if it exists. A reasonable minimum freq value might be 8000.

\membersection{wxMbrola::GetPitch}\label{wxmbrolagetpitch}

\constfunc{float}{GetPitch}{\void}

Gets the current pitch of the voice, between around 0 and 10 (a value of 0.2 is reasonable).

\membersection{wxMbrola::GetVersion}\label{wxmbrolagetversion}

\func{static CString}{GetVersion}{\void}

\func{static wxString}{GetVersion}{\void}

\func{static char*}{GetVersion}{\void}

Returns an Mbrola version string. The third form returns a pointer to a static character
array, which should not be deallocated by the application.

\membersection{wxMbrola::OnRead}\label{wxmbrolaonread}

\func{virtual void}{OnRead}{\void}

Override this function to receive notification when a read operation occurs.

\membersection{wxMbrola::OnStart}\label{wxmbrolaonstart}

\func{virtual void}{OnStart}{\void}

Override this function to receive notification when the sound starts playing.

\membersection{wxMbrola::OnStop}\label{wxmbrolaonstop}

\func{virtual void}{OnStop}{\void}

Override this function to receive notification when the sound stops playing.

\membersection{wxMbrola::OnWait}\label{wxmbrolaonwait}

\func{virtual void}{OnWait}{\void}

Override this function to receive notification when a wait event occurs (don't
know what this actually means).

\membersection{wxMbrola::OnWrite}\label{wxmbrolaonwrite}

\func{virtual void}{OnWrite}{\void}

Override this function to receive notification when a write operation occurs.

\membersection{wxMbrola::Play}\label{wxmbrolaplay}

\func{int}{Play}{\param{const CString\&}{ text}, \param{Bool}{sync}}

\func{int}{Play}{\param{const wxString\&}{ text}, \param{Bool}{sync}}

\func{int}{Play}{\param{const char*}{ text}, \param{Bool}{sync}}

Plays the given phoneme string, synchronously or asynchronously.
The return value is one of the following.

\begin{itemize}\itemsep=0pt
\item wxMBRERR\_NOMBROLADLL
\item wxMBRERR\_NOREGISTRY
\item wxMBRERR\_NORESOURCE
\item wxMBRERR\_NOTHREAD
\item wxMBRERR\_DATABASENOTVALID
\item wxMBRERR\_CANTOPENDEVICEOUT
\item wxMBRERR\_BADCOMMAND
\item wxMBRERR\_CANTOPENFILE
\item wxMBRERR\_WRITEERROR
\item wxMBRERR\_MBROLAERROR
\item wxMBRERR\_CANCELLEDBYUSER
\item wxMBRERR\_NOERROR
\end{itemize}

\membersection{wxMbrola::PlayFromFile}\label{wxmbrolaplayfromfile}

\func{int}{PlayFromFile}{\param{const CString\&}{ filename}, \param{Bool}{sync}}

\func{int}{PlayFromFile}{\param{const wxString\&}{ filename}, \param{Bool}{sync}}

\func{int}{PlayFromFile}{\param{const char*}{ filename}, \param{Bool}{sync}}

Plays the given phoneme file, synchronously or asynchronously.

Please see \helpref{wxMbrola::Play}{wxmbrolaplay} for a description of the return value.

\membersection{wxMbrola::PlayToFile}\label{wxmbrolaplaytofile}

\func{int}{PlayToFile}{\param{const CString\&}{ text}, \param{const CString\&}{ filename}, \param{Bool}{sync}}

\func{int}{PlayToFile}{\param{const wxString\&}{ text}, \param{const wxString\&}{ filename}, \param{Bool}{sync}}

\func{int}{PlayToFile}{\param{const char*}{ text}, \param{const char*}{ filename}, \param{Bool}{sync}}

Plays the given phoneme string to a Windows WAV file, synchronously or asynchronously.

Please see \helpref{wxMbrola::Play}{wxmbrolaplay} for a description of the return value.

\membersection{wxMbrola::SetCurrentDatabase}\label{wxmbrolasetcurrentdatabase}

\func{Bool}{SetCurrentDatabase}{\param{const CString\&}{ db}}

\func{Bool}{SetCurrentDatabase}{\param{const wxString\&}{ db}}

\func{Bool}{SetCurrentDatabase}{\param{const char*}{ db}}

Sets the current diphone database to be used by Mbrola. The name {\it db} is a short label such as "en1".

\membersection{wxMbrola::SetDefaultDatabase}\label{wxmbrolasetdefaultdatabase}

\func{void}{SetDefaultDatabase}{\param{const CString\&}{ db}}

\func{void}{SetDefaultDatabase}{\param{const wxString\&}{ db}}

\func{void}{SetDefaultDatabase}{\param{const char*}{ db}}

Sets the default database. This is not the current database used for synthesis, but the default which
is specified by an optional mbrola.ini file, as a hint.

\membersection{wxMbrola::SetDefaultPath}\label{wxmbrolasetdefaultpath}

\func{void}{SetDefaultPath}{\param{const CString\&}{ path}}

\func{void}{SetDefaultPath}{\param{const wxString\&}{ path}}

\func{void}{SetDefaultPath}{\param{const char*}{ path}}

Sets the default path, as specified in mbrola.ini. Not currently used.

\membersection{wxMbrola::SetDuration}\label{wxmbrolasetduration}

\func{Bool}{SetDuration}{\param{float}{ duration}}

Sets the duration of the voice, between around 0 and 10 (a value of 0.2 is reasonable).

\membersection{wxMbrola::SetFreq}\label{wxmbrolasetfreq}

\func{Bool}{SetFreq}{\param{long}{ freq}}

Sets the frequency of the voice, between around 8000 and 32000. A value of 16000 would be reasonable.

\membersection{wxMbrola::SetMaxPitch}\label{wxmbrolasetmaxpitch}

\func{void}{SetMaxPitch}{\param{float}{ maxPitch}}

Sets the maximum pitch that an application allows the user to have. This value doesn't
affect Mbrola directly, it is up to the application to make use of it. The maximum
pitch is read from the mbrola.ini file if it exists. A reasonable maximum pitch value might be 8.0.

\membersection{wxMbrola::SetMinPitch}\label{wxmbrolasetminpitch}

\func{void}{SetMinPitch}{\param{float}{ minPitch}}

Sets the minimum pitch that an application allows the user to have. This value doesn't
affect Mbrola directly, it is up to the application to make use of it. The minimum
pitch is read from the mbrola.ini file if it exists. A reasonable minimum pitch value might be 0.2.

\membersection{wxMbrola::SetMaxDuration}\label{wxmbrolasetmaxduration}

\func{void}{SetMaxDuration}{\param{float}{ maxDuration}}

Sets the maximum duration that an application allows the user to have. This value doesn't
affect Mbrola directly, it is up to the application to make use of it. The maximum
duration is read from the mbrola.ini file if it exists. A reasonable maximum duration value might be 8.0.

\membersection{wxMbrola::SetMinDuration}\label{wxmbrolasetminduration}

\func{void}{SetMinDuration}{\param{float}{ minDuration}}

Sets the minimum duration that an application allows the user to have. This value doesn't
affect Mbrola directly, it is up to the application to make use of it. The minimum
duration is read from the mbrola.ini file if it exists. A reasonable minimum duration value might be 0.2.

\membersection{wxMbrola::SetMaxFreq}\label{wxmbrolasetmaxfreq}

\func{void}{SetMaxFreq}{\param{long}{ maxFreq}}

Sets the maximum freq that an application allows the user to have. This value doesn't
affect Mbrola directly, it is up to the application to make use of it. The maximum
freq is read from the mbrola.ini file if it exists. A reasonable maximum freq value might be 32000.

\membersection{wxMbrola::SetMinFreq}\label{wxmbrolasetminfreq}

\func{void}{SetMinFreq}{\param{long}{ minFreq}}

Sets the minimum freq that an application allows the user to have. This value doesn't
affect Mbrola directly, it is up to the application to make use of it. The minimum
freq is read from the mbrola.ini file if it exists. A reasonable minimum freq value might be 8000.

\membersection{wxMbrola::SetPitch}\label{wxmbrolasetpitch}

\func{Bool}{SetPitch}{\param{float}{ pitch}}

Sets the pitch of the voice, between around 0 and 10 (a value of 0.2 is reasonable).

\membersection{wxMbrola::Stop}\label{wxmbrolastop}

\func{Bool}{Stop}{\void}

Stops the sound currently playing.

\membersection{wxMbrola::WaitForEnd}\label{wxmbrolawaitforend}

\func{Bool}{WaitForEnd}{\void}

Returns only after the sound has finished playing.

