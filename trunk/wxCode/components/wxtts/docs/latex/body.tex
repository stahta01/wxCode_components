\chapter{Introduction}\label{introduction}
\pagenumbering{arabic}%
\setheader{{\it CHAPTER \thechapter}}{}{}{}{}{{\it CHAPTER \thechapter}}%
\setfooter{\thepage}{}{}{}{}{\thepage}%

This manual describes a small set of classes for text-to-speech generation, currently
under Windows.
The text-to-phoneme part is supplied by modified FreePhone code, mainly written by
Alistair Conkie, Centre for Speech Technology, University of Edinburgh. The
phoneme-to-speech part is performed by Mbrola, a project by Alain Ruelle, Thierry Dutoit and
others at the TCTS Laboratory, Faculte Polytechnique de Mons, Belgium. I am
very grateful to these organisations and individuals for supplying
such useful pieces of software for use by others.

What wxTTS supplies is some C++ wrapper classes and samples to make it relatively easy for
the programmer to use text-to-speech in his or her application. FreePhone
has previously been compiled only under Unix. Mirko Buffoni has contributed
code to replace the Unix-specific phoneme database, enabling me to get FreePhone
running under Windows.

{\it Disclaimer:} Note that I am not a speech specialist, I am merely using my
C++ programming experience to try to get text-to-speech
working under Windows. So I'm likely to be ignorant as anyone about, say, FreePhone internals.
I will probably also make terminological and other blunders. But I would like
to participate in bridging the gap between the excellent research work that's available, and
the reality for Windows programmers. I would also like to cooperate with Unix programmers
in making these classes portable.

For more information, please see below and \helpref{wxTTS classes overview}{ttsoverview}.

This documentation (available in WinHelp, HTML, Word RTF and PDF forms) was generated from Latex
source using Tex2RTF.

\section{Acknowledgements}

FreePhone was mainly written by Alistair Conkie, Centre for Speech Technology, University of Edinburgh.

Mbrola is a project by Alain Ruelle, Thierry Dutoit and others at the TCTS Laboratory,
Faculte Polytechnique de Mons, Belgium. See \urlref{http://tcts.fpms.ac.be/synthesis/mbrola.html}{http://tcts.fpms.ac.be/synthesis/mbrola.html}.

Mirko Buffoni wrote the dictionary code for wxFreePhone.

wxTTS can be downloaded from Julian Smart's Web page at \urlref{http://www.anthemion.co.uk}{http://www.anthemion.co.uk}
 ('Code Cupboard').

\section{General limitations}

Only Windows is supported at present, but this is not a necessary limitation
since Mbrola and FreePhone both work on Unix. Only Visual C++ makefiles (and some project
files) are supplied.

Only English is supported by these classes, since FreePhone is for English only.

FreePhone is not as good at prosody and other aspects as (for example) the Festival
project from the same people, but FreePhone has the advantage of being much smaller
and more easily compiled.

Some words are not pronounced correctly by Mbrola, but Mbrola is being continually improved.

Mbrola puts up a dialog every time it starts, which makes it hard for Mbrola or wxTTS to be used
by blind people for example, but the Mbrola team are considering removing the dialog.

wxFreePhone reads the whole word dictionary into memory (can be around 2MB), which
should not be a restriction for most modern environments.

\section{wxTTS bugs, problems and omissions}

The following work is still needed or can be considered:

\begin{itemize}\itemsep=0pt
\item wxFreePhone needs further testing and debugging for memory problems in cleanup;
\item more error checking and reporting required;
\item (wx)FreePhone code could use rewriting for clarity, perhaps using a string class;
\item need control over how numbers, dates etc. are pronounced;
\item further samples would be useful, for example for reading large texts (which will
need to break text into chunks and wait for a chunk to be read before reading the
next one);
\item custom voices could be maintained in wxMbrola, with a mapping between voice name
and voice parameters;
\item a wxTTS class could be created to unify wxFreePhone and wxMbrola;
\item some SAPI compatibility could be attempted;
\item need to produce standard dialogs that can be invoked to adjust parameters;
\item could have a second (customisable) dictionary, perhaps with a dialog to allow addition of
words in a friendly manner (e.g. you have a selection of phonemes which you can
put together, with a preview - maybe even a soundex-like algorithm to make an initial
guess of what the phonemes should be);
\item need to write further makefiles, e.g. for Gnu-Win32, BC++;
\item need ports to other platforms.
\end{itemize}

See also \helpref{wxFreePhone Discussion}{freephone}, \helpref{wxMbrola Discussion}{mbrola}.

\chapter{Installation}\label{installation}
\setheader{{\it CHAPTER \thechapter}}{}{}{}{}{{\it CHAPTER \thechapter}}%
\setfooter{\thepage}{}{}{}{}{\thepage}%

You will need the Mbrola for Windows installation and the English dictionary, available from
 \urlref{http://tcts.fpms.ac.be/synthesis/mbrola.html}{http://tcts.fpms.ac.be/synthesis/mbrola.html}.

wxTTS can be downloaded from Julian Smart's Web page at \urlref{http://www.anthemion.co.uk}{http://www.anthemion.co.uk}
 ('Code Cupboard').

The wxTTS zip file contains all source code. Unzip into a suitable directory,
taking care to preserve pathnames. The structure is as follows:

{\small\begin{verbatim}
docs/latex/                         ; Latex source for manual
docs/winhelp/                       ; WinHelp version of manual
docs/html/                          ; HTML version of manual
docs/htmlhelp/                      ; Windows HTML Help version of manual
src/freephone/                      ; [wx]FreePhone source
src/freephone/dictionary/           ; Source for dictionary generator, plus english.dat
src/wxmbrola/                       ; Source for wxMbrola
include/tts                         ; Public include files
samples/                            ; Samples
samples/mfc/simple/                 ; Simple MFC demo of wxFreePhone/wxMbrola
samples/mfc/mbrola/                 ; Test of wxMbrola, adapted from the Mbroli utility
samples/win/simple/                 ; Simple plain WIN32 demo of wxFreePhone/wxMbrola
samples/win/freephone/              ; Simple plain WIN32 demo of wxFreePhone only
samples/wxwin/simple/               ; Simple wxWindows demo of wxFreePhone/wxMbrola
lib/                                ; Destination for libraries
lib/mbrplay.lib                     ; Required for all wxTTS apps
dictionary/                         ; Destination of generated FreePhone dictionary files
distrib/                            ; Files for helping to distribute wxTTS
\end{verbatim}
}

Currently, wxTTS only works with VC++ 6.0, but other compilers will be supported at some point.

All libraries and samples may be compiled with the VC++ makefiles (.vc), and some
of the samples may be compiled with supplied project files (.dsp).

The VC++ makefiles can be used to compile three kinds of library, each with Debug and Release
variants. The three kinds are plain WIN32, MFC, and \urlref{wxWindows}{http://www.wxwindows.org}. To use wxTTS with wxWindows
you have to have wxWindows installed and the WXWIN environment variable correctly set.

The following libraries are created and/or used by wxTTS:

{\small\begin{verbatim}
lib/mbrplay.lib                     ; Mbrola import library
lib/wxmbrola.lib                    ; wxMbrola library (WIN32 release)
lib/wxmbrola_d.lib                  ; wxMbrola library (WIN32 debug)
lib/wxmbrola_mfc.lib                ; wxMbrola library (MFC release)
lib/wxmbrola_mfc_d.lib              ; wxMbrola library (MFC debug)
lib/wxmbrola_wx.lib                 ; wxMbrola library (wxWindows release)
lib/wxmbrola_wx_d.lib               ; wxMbrola library (wxWindows debug)
lib/freephone.lib                   ; wxFreePhone library (WIN32 release)
lib/freephone_d.lib                 ; wxFreePhone library (WIN32 debug)
lib/freephone_mfc.lib               ; wxFreePhone library (MFC release)
lib/freephone_mfc_d.lib             ; wxFreePhone library (MFC debug)
lib/freephone_wx.lib                ; wxFreePhone library (wxWindows release)
lib/freephone_wx_d.lib              ; wxFreePhone library (wxWindows debug)
\end{verbatim}
}

wxTTS also generates the following binary dictionary files from english.dat:

{\small\begin{verbatim}
dictionary/english.dct              ; Binary data file
dictionary/english.idx              ; Index file
dictionary/english.key              ; Keywords file
\end{verbatim}
}

These files need to be distributed by any application using wxFreePhone.

\section{Compiling the wxTTS libraries and dictionary}

To compile all the library files, and the dictionary used by wxFreePhone, you can
type this in a DOS box:

{\small\begin{verbatim}
cd src
nmake -f makefile.vc all
\end{verbatim}
}

However, you may wish to only compile some of the libraries, in which case you can use
one or more of these targets:

\begin{itemize}\itemsep=0pt
\item just\_win32: Make WIN32 libraries, debug and release
\item just\_wxwin: Make wxWin libraries, debug and release
\item just\_mfc: Make MFC libraries, debug and release
\item dictionary: Make the dictionary
\end{itemize}

You can also compile the wxMbrola and wxFreePhone libraries individually using the makefile.vc
files provided in each directory under {\tt src}.

Note that all makefiles have a 'clean' target to remove target and intermediate files.

\section{Compiling the wxTTS samples}

To compile all the samples in release mode, you can
type this in a DOS box:

{\small\begin{verbatim}
cd samples
nmake -f makefile.vc all
\end{verbatim}
}

However, you may wish to only compile some of the samples, in which case you can use
one or more of these targets:

\begin{itemize}\itemsep=0pt
\item samples\_win32: Make WIN32 executables, release only
\item samples\_wxwin: Make wxWindows executables, release only
\item samples\_mfc: Make MFC executables, release only
\end{itemize}

You can also compile the samples individually using the makefile.vc
files provided in each directory under {\tt samples}. You can alternatively use the
provided project files if you wish.

Note that all makefiles have a 'clean' target to remove target and intermediate files.

See \helpref{wxTTS Samples}{samples} for a brief description of the samples.

\chapter{wxTTS Samples}\label{samples}
\pagenumbering{arabic}%
\setheader{{\it CHAPTER \thechapter}}{}{}{}{}{{\it CHAPTER \thechapter}}%
\setfooter{\thepage}{}{}{}{}{\thepage}%

\wxheading{WIN32 samples}

{\bf samples/win/simple} shows a dialog that allows you to enter text
and speak it or write it to a phoneme file (test.pho) or a wave file (test.wav).

{\bf samples/win/freephone} only uses wxFreePhone, and allows you to translate text to
a phoneme file (test.pho).

\wxheading{MFC samples}

{\bf samples/win/simple} shows a dialog that allows you to enter text
and speak it or write it to a phoneme file (test.pho) or a wave file (test.wav).

{\bf samples/win/mbrola} is an adaptation of code supplied by the authors of Mbrola,
and allows setting of various Mbrola parameters.

\wxheading{wxWindows samples}

{\bf samples/win/simple} shows a dialog that allows you to enter text
and play it to the sound card or a wave file. You can also play a phoneme file.

\chapter{wxFreePhone Discussion}\label{freephone}
\setheader{{\it CHAPTER \thechapter}}{}{}{}{}{{\it CHAPTER \thechapter}}%
\setfooter{\thepage}{}{}{}{}{\thepage}%

Here are some random observations about (wx)FreePhone. Note that this hasn't been updated
for a long time (since June 1998) and therefore improvements to Mbrola may make some of the comments
redundant.

\wxheading{Implementation}

I have taken the Unix FreePhone code, renamed the .c files to .cpp, and made miscellaneous
changes to make it compile using Visual C++ 5.0. freephone.h, freephone.cpp adds a class
wrapper around the main driving bits of the code. The major stumbling block
was the dictionary database, which uses a library not available under Windows.
Mirko Buffoni has come to the rescue with his MFDict.h, MFDict.cpp code which
implements a simple Dictionary class for searching prepared files. There is also
MkDict.exe which generates these files from the Edinburgh phoneme database file. The output
files are: a keyword file (.key), and index file (.idx), and a data file (.dct).
Mirko's code reads the data into memory which may be a constraint for some people, but
these days megabytes are abundant. Although the code does some linear searching, it seems quite fast,
which can be verified by writing out .pho files using samples/win/simple/simple.exe.
A message box indicates when the process is complete, and this is normally pretty instant
(at least it is on my PII 300MHz).

\wxheading{Dictionary files}

The English dictionary source file is in freephone/dictionary. The output files,
with extension key, idx and dct, are put into the top-level dictionary directory
in the wxTTS installation. These are used by the samples, which specify where these
files are in the wxFreePhone constructor. If the files are not found, the accuracy
of the speech will be severely degraded as FreePhone guesses the pronunciations.

\wxheading{Bad phonemes}

If it does not find a word in its dictionary (and sometimes even when it does), FreePhone sometimes phonemes
which cannot be processed by Mbrola. Perhaps we need a program to run through the dictionary pronouncing all the words, so
we can get an idea about how much of it works, and what the problem areas are.

\wxheading{Memory bugs}

There are memory bugs in FreePhone, which may or may not be due to things I have
done. Sometimes the bugs are fatal, sometimes they just cause warnings in Developer
Studio.

\wxheading{Rewriting (wx)FreePhone}

It will be tricky to rewrite FreePhone, but I think we probably need to
do this, so that we can debug and extend it. A written analysis of FreePhone is required.
Anyone know any good C analysis/documentation tools?

Some suggestions for the rewrite are as follows:

\begin{itemize}\itemsep=0pt
\item Have a ttsString class which we use extensively, instead of char*. This
string class can be taken from the wxWindows 2.0 wxString class, and optionally #defined
to be CString or wxString if MFC or wxWindows is available.
\item Implement most of FreePhone with a FreePhone class, which is used by wxFreePhone.
The separation means that we don't have to include all implementation functions and data
into wxFreePhone.
\item Gradually replace the C code with C++ code, starting from the top level and working
towards the lower-level functions.
\end{itemize}

\wxheading{Porting}

It should be quite straightforward to get my version of FreePhone compiled on Unix.
Either the original database access routines can be used, or Mirko Buffoni's MFDict.cpp.
It probably makes sense to stick with MFDict.cpp on all platforms for simplicity, and
its performance seems quite adequate.

\chapter{wxMbrola Discussion}\label{mbrola}
\setheader{{\it CHAPTER \thechapter}}{}{}{}{}{{\it CHAPTER \thechapter}}%
\setfooter{\thepage}{}{}{}{}{\thepage}%

\wxheading{mbrola.ini file format}

wxMbrola tries to find an mbrola.ini file in the current directory or Windows directory.
This file allows the application to specify useful values, which don't affect wxMbrola
directly but can be used in an application to tailor the voice parameters, etc. This
facility might be considered too application-specific to be in wxMbrola and could be
removed in the future.

\begin{verbatim}
[GENERAL]
Path=<string> ; Specifies a path (what for?)

[PITCH]
Min=<float>   ; E.g. 0.2
Max=<float>   ; E.g. 8.0

[TIME]
Min=<float>   ; E.g. 0.2
Max=<float>   ; E.g. 8.0

[VOICE] ; Frequency. What about having default freq. (16000)?
Min=<float>   ; E.g. 8000
Max=<float>   ; E.g. 32000

[DATABASE]
Default=<string>

\end{verbatim}

