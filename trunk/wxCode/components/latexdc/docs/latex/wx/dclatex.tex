% CSH \section{\class{wxLatexDC}}\label{wxlatexdc}

\section{wxLatexDC}

wxLatexDC is a device context that will generate a LaTeX2e file from graphics and text that
were drawn onto it. It makes use of PSTricks (http://tug.org/PSTricks) to create Tex-compatible
PostScript macros. I've shamelessly copied large parts of the code from the wxSVGFileDC and
wxPostScriptDC classes. You can create a wxLatexDC in exactly the same way as a wxSVGFileDC.
At present, only a small part of the DC drawing functions have been implemented. 
The major advantage of using LaTeX output rather than directly writing PostScript is that 
typesetting can be done in LaTeX, so that graphics seamlessly integrate into documents. A 
positive side effect is that the PSTricks macros are more easily readable than plain PostScript.
The obvious disadvantage is that wxLatexDC can't know which font type and size will be used in
the final document, so that all the functions that try to retrieve information about text size
will not work (see comment in \helpref{GetCharHeight}{wxlatexdcgetcharheight}). To still be able to
position text precisely at the location that you want to have it in your final LaTeX document, you 
can make use of \helpref{wxLatexDC::DrawLabelLatex}{wxlatexdcdrawlabellatex} which uses the same syntax as 
wxDC::DrawLabel(). This allows you to set your text alignment (left, right, top, bottom, horizontal 
and vertical center) relative to the wxRect that is passed as an argument.

\wxheading{Derived from}

wxDC \\ \\
% CSH \helpref{wxDC}{wxlatexdc}

\wxheading{Include files}

<wx/dclatex.h>

\wxheading{Library}

None (wxCode) \\ \\
% CSH \helpref{wxCore}{librarieslist}

% CSH \wxheading{See also}

%\helpref{Overview}{dcoverview}


\latexignore{\rtfignore{\wxheading{Members}}}

\membersection{wxLatexDC::wxLatexDC}\label{wxlatexdcctor}

Constructors:

\func{}{wxLatexDC}{\param{wxString}{ filename}}

Creates a file (\it filename) with default size 340x240 at 72.0 dots per inch (a frequent screen resolution).

\func{}{wxLatexDC}{\param{wxString}{ filename}, \param{int}{ Width},\param{int}{ Height}}

Creates a file (\it filename) with size {\it Width} by {\it Height} at 72.0 dots per inch 

\func{}{wxLatexDC}{\param{wxString}{ filename}, \param{int}{ Width},\param{int}{ Height},\param{float}{ dpi}}

Creates a file (\it filename) with size {\it Width} by {\it Height} at {\it dpi} resolution.

\membersection{wxLatexDC::\destruct{wxLatexDC}}\label{wxlatexdcdtor}

\func{}{\destruct{wxLatexDC}}{\void}

Destructor.

\membersection{wxLatexDC::BeginDrawing}\label{wxlatexdcbegindrawing}

Does nothing

\membersection{wxLatexDC::Blit}\label{wxlatexdcblit}

\func{bool}{Blit}{\param{wxCoord}{ xdest}, \param{wxCoord}{ ydest}, \param{wxCoord}{ width}, \param{wxCoord}{ height},
  \param{wxLatexDC* }{source}, \param{wxCoord}{ xsrc}, \param{wxCoord}{ ysrc}, \param{int}{ logicalFunc = wxCOPY},
  \param{bool }{useMask = FALSE}, \param{wxCoord}{ xsrcMask = -1}, \param{wxCoord}{ ysrcMask = -1}}

% As wxDC: Copy from a source DC to this DC, specifying the destination
% coordinates, size of area to copy, source DC, source coordinates,
% logical function, whether to use a bitmap mask, and mask source position.

Not implemented yet.

\membersection{wxLatexDC::CalcBoundingBox}\label{wxlatexdccalcboundingbox}

\func{void}{CalcBoundingBox}{\param{wxCoord }{x}, \param{wxCoord }{y}}

Adds the specified point to the bounding box which can be retrieved with 
\helpref{MinX}{wxlatexdcminx}, \helpref{MaxX}{wxlatexdcmaxx} and 
\helpref{MinY}{wxlatexdcminy}, \helpref{MaxY}{wxlatexdcmaxy} functions.

Note that the size and type of the fonts that will be used in the final
document are not known when a wxLatexDC is created. Therefore, the extent
of any text in the drawing will not be accounted for when the bounding
box is calculated.
 
\membersection{wxLatexDC::Clear}\label{wxlatexdcclear}

\func{void}{Clear}{\void}

Does nothing.


\membersection{wxLatexDC::CrossHair}\label{wxlatexdccrosshair}

\func{void}{CrossHair}{\param{wxCoord}{ x}, \param{wxCoord}{ y}}

Not Implemented

\membersection{wxLatexDC::DestroyClippingRegion}\label{wxlatexdcdestroyclippingregion}

\func{void}{DestroyClippingRegion}{\void}

Not Implemented

\membersection{wxLatexDC::DeviceToLogicalX}\label{wxlatexdcdevicetologicalx}

\func{wxCoord}{DeviceToLogicalX}{\param{wxCoord}{ x}}

Convert device X coordinate to logical coordinate, using the current
mapping mode.

\membersection{wxLatexDC::DeviceToLogicalXRel}\label{wxlatexdcdevicetologicalxrel}

\func{wxCoord}{DeviceToLogicalXRel}{\param{wxCoord}{ x}}

Convert device X coordinate to relative logical coordinate, using the current
mapping mode but ignoring the x axis orientation.
Use this function for converting a width, for example.

\membersection{wxLatexDC::DeviceToLogicalY}\label{wxlatexdcdevicetologicaly}

\func{wxCoord}{DeviceToLogicalY}{\param{wxCoord}{ y}}

Converts device Y coordinate to logical coordinate, using the current
mapping mode.

\membersection{wxLatexDC::DeviceToLogicalYRel}\label{wxlatexdcdevicetologicalyrel}

\func{wxCoord}{DeviceToLogicalYRel}{\param{wxCoord}{ y}}

Convert device Y coordinate to relative logical coordinate, using the current
mapping mode but ignoring the y axis orientation.
Use this function for converting a height, for example.

\membersection{wxLatexDC::DrawArc}\label{wxlatexdcdrawarc}

\func{void}{DrawArc}{\param{wxCoord}{ x1}, \param{wxCoord}{ y1}, \param{wxCoord}{ x2}, \param{wxCoord}{ y2}, \param{wxCoord}{ xc}, \param{wxCoord}{ yc}}

Draws an arc of a circle, centred on ({\it xc, yc}), with starting point ({\it x1, y1})
and ending at ({\it x2, y2}).   The current pen is used for the outline
and the current brush for filling the shape.

The arc is drawn in an anticlockwise direction from the start point to the end point.

\membersection{wxLatexDC::DrawBitmap}\label{wxlatexdcdrawbitmap}

\func{void}{DrawBitmap}{\param{const wxBitmap\&}{ bitmap}, \param{wxCoord}{ x}, \param{wxCoord}{ y}, \param{bool}{ transparent}}

%Draw a bitmap on the device context at the specified point. If {\it transparent} is true and the bitmap has
%a transparency mask, the bitmap will be drawn transparently.

%When drawing a mono-bitmap, the current text foreground colour will be used to draw the foreground
%of the bitmap (all bits set to 1), and the current text background colour to draw the background
%(all bits set to 0). See also \helpref{SetTextForeground}{wxlatexdcsettextforeground}, 

% \helpref{SetTextBackground}{wxlatexdcsettextbackground} and \helpref{wxMemoryDC}{wxmemorydc}.

Not implemented. Would require converting the bitmap into an eps, I believe.
Feel free to implement that.

\membersection{wxLatexDC::DrawCheckMark}\label{wxlatexdcdrawcheckmark}

\func{void}{DrawCheckMark}{\param{wxCoord}{ x}, \param{wxCoord}{ y}, \param{wxCoord}{ width}, \param{wxCoord}{ height}}

\func{void}{DrawCheckMark}{\param{const wxRect \&}{rect}}

Draws a check mark inside the given rectangle.

\membersection{wxLatexDC::DrawCircle}\label{wxlatexdcdrawcircle}

\func{void}{DrawCircle}{\param{wxCoord}{ x}, \param{wxCoord}{ y}, \param{wxCoord}{ radius}}

\func{void}{DrawCircle}{\param{const wxPoint\&}{ pt}, \param{wxCoord}{ radius}}

Draws a circle with the given centre and radius.

\wxheading{See also}

\helpref{DrawEllipse}{wxlatexdcdrawellipse}

\membersection{wxLatexDC::DrawEllipse}\label{wxlatexdcdrawellipse}

\func{void}{DrawEllipse}{\param{wxCoord}{ x}, \param{wxCoord}{ y}, \param{wxCoord}{ width}, \param{wxCoord}{ height}}

\func{void}{DrawEllipse}{\param{const wxPoint\&}{ pt}, \param{const wxSize\&}{ size}}

\func{void}{DrawEllipse}{\param{const wxRect\&}{ rect}}

Draws an ellipse contained in the rectangle specified either with the given top
left corner and the given size or directly. The current pen is used for the
outline and the current brush for filling the shape.

\wxheading{See also}

\helpref{DrawCircle}{wxlatexdcdrawcircle}

\membersection{wxLatexDC::DrawEllipticArc}\label{wxlatexdcdrawellipticarc}

\func{void}{DrawEllipticArc}{\param{wxCoord}{ x}, \param{wxCoord}{ y}, \param{wxCoord}{ width}, \param{wxCoord}{ height},
 \param{double}{ start}, \param{double}{ end}}

Draws an arc of an ellipse. The current pen is used for drawing the arc and
the current brush is used for drawing the pie.

{\it x} and {\it y} specify the x and y coordinates of the upper-left corner of the rectangle that contains
the ellipse.

{\it width} and {\it height} specify the width and height of the rectangle that contains
the ellipse.

{\it start} and {\it end} specify the start and end of the arc relative to the three-o'clock
position from the center of the rectangle. Angles are specified
in degrees (360 is a complete circle). Positive values mean
counter-clockwise motion. If {\it start} is equal to {\it end}, a
complete ellipse will be drawn.

\membersection{wxLatexDC::DrawIcon}\label{wxlatexdcdrawicon}

\func{void}{DrawIcon}{\param{const wxIcon\&}{ icon}, \param{wxCoord}{ x}, \param{wxCoord}{ y}}

% Draw an icon on the display (does nothing if the device context is PostScript).
% This can be the simplest way of drawing bitmaps on a window.
Not implemented yet. See comment in \helpref{DrawBitmap}{wxlatexdcdrawbitmap}.

\membersection{wxLatexDC::DrawLabelLatex}\label{wxlatexdcdrawlabellatex}

\func{void}{DrawLabelLatex}{\param{const wxString\&}{ text}, \param{const wxRect\&}{ rect}, \param{int}{ alignment = \texttt{wxALIGN\_LEFT \| wxALIGN\_TOP}}, \param{int}{ indexAccel = -1}}

Draws text with the specified \it alignment relative to the \it rect. The rationale for having this function is
that you can precisely position the \it rect on your drawing, and then align your text using this rectangle as
a reference. The alignment will be kept even in your final document, independent of which font you will use in
LaTeX.
I couldn't find a way of overriding wxDC::DrawLabel to do exactly that, please notify me if you find a solution.

\wxheading{See also}

\helpref{DrawText}{wxlatexdcdrawtext}, \helpref{DrawRotatedText}{wxlatexdcdrawrotatedtext}

\membersection{wxLatexDC::DrawLine}\label{wxlatexdcdrawline}

\func{void}{DrawLine}{\param{wxCoord}{ x1}, \param{wxCoord}{ y1}, \param{wxCoord}{ x2}, \param{wxCoord}{ y2}}

Draws a line from the first point to the second. The current pen is used
for drawing the line.

\membersection{wxLatexDC::DrawLines}\label{wxlatexdcdrawlines}

\func{void}{DrawLines}{\param{int}{ n}, \param{wxPoint}{ points[]}, \param{wxCoord}{ xoffset = 0}, \param{wxCoord}{ yoffset = 0}}

\func{void}{DrawLines}{\param{wxList *}{points}, \param{wxCoord}{ xoffset = 0}, \param{wxCoord}{ yoffset = 0}}

Draws lines using an array of {\it points} of size {\it n}, or list of
pointers to points, adding the optional offset coordinate. The current
pen is used for drawing the lines.  The programmer is responsible for
deleting the list of points.

\membersection{wxLatexDC::DrawPolygon}\label{wxlatexdcdrawpolygon}

\func{void}{DrawPolygon}{\param{int}{ n}, \param{wxPoint}{ points[]}, \param{wxCoord}{ xoffset = 0}, \param{wxCoord}{ yoffset = 0},\\
  \param{int }{fill\_style = wxODDEVEN\_RULE}}

\func{void}{DrawPolygon}{\param{wxList *}{points}, \param{wxCoord}{ xoffset = 0}, \param{wxCoord}{ yoffset = 0},\\
  \param{int }{fill\_style = wxODDEVEN\_RULE}}

Draws a filled polygon using an array of {\it points} of size {\it n},
or list of pointers to points, adding the optional offset coordinate.

The last argument specifies the fill rule: {\bf wxODDEVEN\_RULE} (the
default) or {\bf wxWINDING\_RULE}.

The current pen is used for drawing the outline, and the current brush
for filling the shape.  Using a transparent brush suppresses filling.
The programmer is responsible for deleting the list of points.

Note that wxWindows automatically closes the first and last points.

Only few filling styles have been implemented yet.


\membersection{wxLatexDC::DrawPoint}\label{wxlatexdcdrawpoint}

\func{void}{DrawPoint}{\param{wxCoord}{ x}, \param{wxCoord}{ y}}

Draws a point using the current pen.

\membersection{wxLatexDC::DrawRectangle}\label{wxlatexdcdrawrectangle}

\func{void}{DrawRectangle}{\param{wxCoord}{ x}, \param{wxCoord}{ y}, \param{wxCoord}{ width}, \param{wxCoord}{ height}}

Draws a rectangle with the given top left corner, and with the given
size.  The current pen is used for the outline and the current brush
for filling the shape.

\membersection{wxLatexDC::DrawRotatedText}\label{wxlatexdcdrawrotatedtext}

\func{void}{DrawRotatedText}{\param{const wxString\& }{text}, \param{wxCoord}{ x}, \param{wxCoord}{ y}, \param{double}{ angle}}

Draws the text rotated by {\it angle} degrees. See comments in \helpref{DrawLabelLatex}{wxlatexdcdrawlabellatex}, 
\helpref{CalcBoundingBox}{wxlatexdccalcboundingbox} and \helpref{DrawText}{wxlatexdcdrawtext} if you want
to have better control of text positioning in your final LaTeX document.

%The wxMSW wxDC and wxLatexDC rotate the text around slightly different points, depending on the size of the font
\wxheading{See also}

\helpref{DrawLabelLatex}{wxlatexdcdrawlabellatex}, \helpref{DrawText}{wxlatexdcdrawtext}

\membersection{wxLatexDC::DrawRoundedRectangle}\label{wxlatexdcdrawroundedrectangle}

\func{void}{DrawRoundedRectangle}{\param{wxCoord}{ x}, \param{wxCoord}{ y}, \param{wxCoord}{ width}, \param{wxCoord}{ height}, \param{double}{ radius = 20}}

Draws a rectangle with the given top left corner, and with the given
size.  The corners are quarter-circles using the given radius. The
current pen is used for the outline and the current brush for filling
the shape.

If {\it radius} is positive, the value is assumed to be the
radius of the rounded corner. If {\it radius} is negative,
the absolute value is assumed to be the {\it proportion} of the smallest
dimension of the rectangle. This means that the corner can be
a sensible size relative to the size of the rectangle, and also avoids
the strange effects X produces when the corners are too big for
the rectangle.

\membersection{wxLatexDC::DrawSpline}\label{wxlatexdcdrawspline}

\func{void}{DrawSpline}{\param{wxList *}{points}}

% Draws a spline between all given control points, using the current
% pen.  Doesn't delete the wxList and contents. The spline is drawn
% using a series of lines, using an algorithm taken from the X drawing
% program `XFIG'.

This uses the base class implementation to draw a spline, which
doesn't give nice results at present. 

\func{void}{DrawSpline}{\param{wxCoord}{ x1}, \param{wxCoord}{ y1}, \param{wxCoord}{ x2}, \param{wxCoord}{ y2}, \param{wxCoord}{ x3}, \param{wxCoord}{ y3}}

% Draws a three-point spline using the current pen.
This uses the base class implementation to draw a spline, which
doesn't give nice results at present. 

\membersection{wxLatexDC::DrawText}\label{wxlatexdcdrawtext}

\func{void}{DrawText}{\param{const wxString\& }{text}, \param{wxCoord}{ x}, \param{wxCoord}{ y}}

Draws a text string at the specified point, using the current text font,
and the current text foreground and background colours.

The coordinates refer to the top-left corner of the rectangle bounding
the string. Note that wxLatexDC has no idea which font you will use in
your final LaTeX document, so there is no way how to find out the extent
of the text. See \helpref{DrawLabelLatex}{wxlatexdcdrawlabellatex}
if you want to have better control of text positioning in your final LaTeX document.

\wxheading{See also}

\helpref{DrawLabelLatex}{wxlatexdcdrawlabellatex}, \helpref{DrawRotatedText}{wxlatexdcdrawrotatedtext}

\membersection{wxLatexDC::EndDoc}\label{wxlatexdcenddoc}

\func{void}{EndDoc}{\void}

Does nothing

\membersection{wxLatexDC::EndDrawing}\label{wxlatexdcenddrawing}

\func{void}{EndDrawing}{\void}

Does nothing

\membersection{wxLatexDC::EndPage}\label{wxlatexdcendpage}

\func{void}{EndPage}{\void}

Does nothing

\membersection{wxLatexDC::FloodFill}\label{wxlatexdcfloodfill}

\func{void}{FloodFill}{\param{wxCoord}{ x}, \param{wxCoord}{ y}, \param{const wxColour\&}{ colour}, \param{int}{ style=wxFLOOD\_SURFACE}}

Not implemented

\membersection{wxLatexDC::GetBackground}\label{wxlatexdcgetbackground}

\func{wxBrush\&}{GetBackground}{\void}

\constfunc{const wxBrush\&}{GetBackground}{\void}

% Gets the brush used for painting the background (see \helpref{wxLatexDC::SetBackground}{wxlatexdcsetbackground}).
Not implemented.

\membersection{wxLatexDC::GetBackgroundMode}\label{wxlatexdcgetbackgroundmode}

\constfunc{int}{GetBackgroundMode}{\void}

% Returns the current background mode: {\tt wxSOLID} or {\tt wxTRANSPARENT}.
Not implemented.

\wxheading{See also}

\helpref{SetBackgroundMode}{wxlatexdcsetbackgroundmode}

\membersection{wxLatexDC::GetBrush}\label{wxlatexdcgetbrush}

\func{wxBrush\&}{GetBrush}{\void}

\constfunc{const wxBrush\&}{GetBrush}{\void}

Gets the current brush (see \helpref{wxLatexDC::SetBrush}{wxlatexdcsetbrush}).

\membersection{wxLatexDC::GetCharHeight}\label{wxlatexdcgetcharheight}

\func{wxCoord}{GetCharHeight}{\void}

Not implemented:
Since wxLatexDC has no idea which font you will use in
your final LaTeX document, there is no way to find out anything
about character size and font style.

\membersection{wxLatexDC::GetCharWidth}\label{wxlatexdcgetcharwidth}

\func{wxCoord}{GetCharWidth}{\void}

Not implemented: See \helpref{GetCharHeight}{wxlatexdcgetcharheight} for reasons.

\membersection{wxLatexDC::GetClippingBox}\label{wxlatexdcgetclippingbox}

\func{void}{GetClippingBox}{\param{wxCoord}{ *x}, \param{wxCoord}{ *y}, \param{wxCoord}{ *width}, \param{wxCoord}{ *height}}

Not implemented

\membersection{wxLatexDC::GetFont}\label{wxlatexdcgetfont}

\func{wxFont\&}{GetFont}{\void}

\constfunc{const wxFont\&}{GetFont}{\void}

Not implemented: See \helpref{GetCharHeight}{wxlatexdcgetcharheight} for reasons.

\membersection{wxLatexDC::GetLogicalFunction}\label{wxlatexdcgetlogicalfunction}

\func{int}{GetLogicalFunction}{\void}

Gets the current logical function (see \helpref{wxLatexDC::SetLogicalFunction}{wxlatexdcsetlogicalfunction}).

\membersection{wxLatexDC::GetMapMode}\label{wxlatexdcgetmapmode}

\func{int}{GetMapMode}{\void}

Gets the {\it mapping mode} for the device context (see \helpref{wxLatexDC::SetMapMode}{wxlatexdcsetmapmode}).

\membersection{wxLatexDC::GetPen}\label{wxlatexdcgetpen}

\func{wxPen\&}{GetPen}{\void}

\constfunc{const wxPen\&}{GetPen}{\void}

Gets the current pen (see \helpref{wxLatexDC::SetPen}{wxlatexdcsetpen}).

\membersection{wxLatexDC::GetPixel}\label{wxlatexdcgetpixel}

\func{bool}{GetPixel}{\param{wxCoord}{ x}, \param{wxCoord}{ y}, \param{wxColour *}{colour}}

Not implemented

\membersection{wxLatexDC::GetSize}\label{wxlatexdcgetsize}

\func{void}{GetSize}{\param{wxCoord *}{width}, \param{wxCoord *}{height}}


For a Windows printer device context, this gets the horizontal and vertical
resolution. 

\membersection{wxLatexDC::GetTextBackground}\label{wxlatexdcgettextbackground}

\func{wxColour\&}{GetTextBackground}{\void}

\constfunc{const wxColour\&}{GetTextBackground}{\void}

Gets the current text background colour (see \helpref{wxLatexDC::SetTextBackground}{wxlatexdcsettextbackground}).

\membersection{wxLatexDC::GetTextExtent}\label{wxlatexdcgettextextent}

\func{void}{GetTextExtent}{\param{const wxString\& }{string}, \param{wxCoord *}{w}, \param{wxCoord *}{h},\\
  \param{wxCoord *}{descent = NULL}, \param{wxCoord *}{externalLeading = NULL}, \param{wxFont *}{font = NULL}}

Not implemented: See \helpref{GetCharHeight}{wxlatexdcgetcharheight} for reasons.

See also \helpref{wxLatexDC::SetFont}{wxlatexdcsetfont}.
% CSH \helpref{wxFont}{wxfont}, 

\membersection{wxLatexDC::GetTextForeground}\label{wxlatexdcgettextforeground}

\func{wxColour\&}{GetTextForeground}{\void}

\constfunc{const wxColour\&}{GetTextForeground}{\void}

Gets the current text foreground colour (see \helpref{wxLatexDC::SetTextForeground}{wxlatexdcsettextforeground}).


\membersection{wxLatexDC::GetUserScale}\label{wxlatexdcgetuserscale}

\func{void}{GetUserScale}{\param{double}{ *x}, \param{double}{ *y}}

Gets the current user scale factor (set by \helpref{SetUserScale}{wxlatexdcsetuserscale}).

\membersection{wxLatexDC::LogicalToDeviceX}\label{wxlatexdclogicaltodevicex}

\func{wxCoord}{LogicalToDeviceX}{\param{wxCoord}{ x}}

Converts logical X coordinate to device coordinate, using the current
mapping mode.

\membersection{wxLatexDC::LogicalToDeviceXRel}\label{wxlatexdclogicaltodevicexrel}

\func{wxCoord}{LogicalToDeviceXRel}{\param{wxCoord}{ x}}

Converts logical X coordinate to relative device coordinate, using the current
mapping mode but ignoring the x axis orientation.
Use this for converting a width, for example.

\membersection{wxLatexDC::LogicalToDeviceY}\label{wxlatexdclogicaltodevicey}

\func{wxCoord}{LogicalToDeviceY}{\param{wxCoord}{ y}}

Converts logical Y coordinate to device coordinate, using the current
mapping mode.

\membersection{wxLatexDC::LogicalToDeviceYRel}\label{wxlatexdclogicaltodeviceyrel}

\func{wxCoord}{LogicalToDeviceYRel}{\param{wxCoord}{ y}}

Converts logical Y coordinate to relative device coordinate, using the current
mapping mode but ignoring the y axis orientation.
Use this for converting a height, for example.

\membersection{wxLatexDC::MaxX}\label{wxlatexdcmaxx}

\func{wxCoord}{MaxX}{\void}

Gets the maximum horizontal extent used in drawing commands so far.

\membersection{wxLatexDC::MaxY}\label{wxlatexdcmaxy}

\func{wxCoord}{MaxY}{\void}

Gets the maximum vertical extent used in drawing commands so far.

\membersection{wxLatexDC::MinX}\label{wxlatexdcminx}

\func{wxCoord}{MinX}{\void}

Gets the minimum horizontal extent used in drawing commands so far.

\membersection{wxLatexDC::MinY}\label{wxlatexdcminy}

\func{wxCoord}{MinY}{\void}

Gets the minimum vertical extent used in drawing commands so far.

\membersection{wxLatexDC::Ok}\label{wxlatexdcok}

\func{bool}{Ok}{\void}

Returns true if the DC is ok to use; False values arise from being unable to 
write the file

\membersection{wxLatexDC::ResetBoundingBox}\label{wxlatexdcresetboundingbox}

\func{void}{ResetBoundingBox}{\void}

Resets the bounding box: after a call to this function, the bounding box
doesn't contain anything.

\wxheading{See also}

\helpref{CalcBoundingBox}{wxlatexdccalcboundingbox}

\membersection{wxLatexDC::SetAxisOrientation}\label{wxlatexdcsetaxisorientation}

\func{void}{SetAxisOrientation}{\param{bool}{ xLeftRight},
                                \param{bool}{ yBottomUp}}

Sets the x and y axis orientation (i.e., the direction from lowest to
highest values on the axis). The default orientation is the natural
orientation, e.g. x axis from left to right and y axis from bottom up.

\wxheading{Parameters}

\docparam{xLeftRight}{True to set the x axis orientation to the natural
left to right orientation, false to invert it.}

\docparam{yBottomUp}{True to set the y axis orientation to the natural
bottom up orientation, false to invert it.}

\membersection{wxLatexDC::SetDeviceOrigin}\label{wxlatexdcsetdeviceorigin}

\func{void}{SetDeviceOrigin}{\param{wxCoord}{ x}, \param{wxCoord}{ y}}

Sets the device origin (i.e., the origin in pixels after scaling has been
applied).

This function may be useful in Windows printing
operations for placing a graphic on a page.

\membersection{wxLatexDC::SetBackground}\label{wxlatexdcsetbackground}

\func{void}{SetBackground}{\param{const wxBrush\& }{brush}}

Sets the current background brush for the DC.

\membersection{wxLatexDC::SetBackgroundMode}\label{wxlatexdcsetbackgroundmode}

\func{void}{SetBackgroundMode}{\param{int}{ mode}}

{\it mode} may be one of wxSOLID and wxTRANSPARENT. This setting determines
whether text will be drawn with a background colour or not.

\membersection{wxLatexDC::SetClippingRegion}\label{wxlatexdcsetclippingregion}

\func{void}{SetClippingRegion}{\param{wxCoord}{ x}, \param{wxCoord}{ y}, \param{wxCoord}{ width}, \param{wxCoord}{ height}}

\func{void}{SetClippingRegion}{\param{const wxPoint\& }{pt}, \param{const wxSize\& }{sz}}

\func{void}{SetClippingRegion}{\param{const wxRect\&}{ rect}}

\func{void}{SetClippingRegion}{\param{const wxRegion\&}{ region}}

Not implemented 


\membersection{wxLatexDC::SetPalette}\label{wxlatexdcsetpalette}

\func{void}{SetPalette}{\param{const wxPalette\& }{palette}}

Not implemented 

\membersection{wxLatexDC::SetBrush}\label{wxlatexdcsetbrush}

\func{void}{SetBrush}{\param{const wxBrush\& }{brush}}

Sets the current brush for the DC.

If the argument is wxNullBrush, the current brush is selected out of the device
context, and the original brush restored, allowing the current brush to
be destroyed safely.

% See also \helpref{wxBrush}{wxbrush}.

% See also \helpref{wxMemoryDC}{wxmemorydc} for the interpretation of colours
% when drawing into a monochrome bitmap.

\membersection{wxLatexDC::SetFont}\label{wxlatexdcsetfont}

\func{void}{SetFont}{\param{const wxFont\& }{font}}

Not implemented: See \helpref{GetCharHeight}{wxlatexdcgetcharheight} for reasons.

% CSH See also \helpref{wxFont}{wxfont}.

\membersection{wxLatexDC::SetLogicalFunction}\label{wxlatexdcsetlogicalfunction}

\func{void}{SetLogicalFunction}{\param{int}{ function}}


Only wxCOPY is avalaible; trying to set one of the other values will fail

\membersection{wxLatexDC::SetMapMode}\label{wxlatexdcsetmapmode}

\func{void}{SetMapMode}{\param{int}{ int}}

The {\it mapping mode} of the device context defines the unit of
measurement used to convert logical units to device units. Note that
in X, text drawing isn't handled consistently with the mapping mode; a
font is always specified in point size. However, setting the {\it
user scale} (see \helpref{wxLatexDC::SetUserScale}{wxlatexdcsetuserscale}) scales the text appropriately. In
Windows, scaleable TrueType fonts are always used; in X, results depend
on availability of fonts, but usually a reasonable match is found.

Note that the coordinate origin should ideally be selectable, but for
now is always at the top left of the screen/printer.

Drawing to a Windows printer device context under UNIX
uses the current mapping mode, but mapping mode is currently ignored for
PostScript output.

The mapping mode can be one of the following:

\begin{twocollist}\itemsep=0pt
\twocolitem{wxMM\_TWIPS}{Each logical unit is 1/20 of a point, or 1/1440 of
  an inch.}
\twocolitem{wxMM\_POINTS}{Each logical unit is a point, or 1/72 of an inch.}
\twocolitem{wxMM\_METRIC}{Each logical unit is 1 mm.}
\twocolitem{wxMM\_LOMETRIC}{Each logical unit is 1/10 of a mm.}
\twocolitem{wxMM\_TEXT}{Each logical unit is 1 pixel.}
\end{twocollist}

\membersection{wxLatexDC::SetPen}\label{wxlatexdcsetpen}

\func{void}{SetPen}{\param{const wxPen\& }{pen}}

Sets the current pen for the DC.

If the argument is wxNullPen, the current pen is selected out of the device
context, and the original pen restored.

% See also \helpref{wxMemoryDC}{wxmemorydc} for the interpretation of colours
% when drawing into a monochrome bitmap.

\membersection{wxLatexDC::SetTextBackground}\label{wxlatexdcsettextbackground}

\func{void}{SetTextBackground}{\param{const wxColour\& }{colour}}

Sets the current text background colour for the DC.

\membersection{wxLatexDC::SetTextForeground}\label{wxlatexdcsettextforeground}

\func{void}{SetTextForeground}{\param{const wxColour\& }{colour}}

Sets the current text foreground colour for the DC.

% See also \helpref{wxMemoryDC}{wxmemorydc} for the interpretation of colours
% when drawing into a monochrome bitmap.

\membersection{wxLatexDC::SetUserScale}\label{wxlatexdcsetuserscale}

\func{void}{SetUserScale}{\param{double}{ xScale}, \param{double}{ yScale}}

Sets the user scaling factor, useful for applications which require
`zooming'.

\membersection{wxLatexDC::StartDoc}\label{wxlatexdcstartdoc}

\func{bool}{StartDoc}{\param{const wxString\& }{message}}

Does nothing

\membersection{wxLatexDC::StartPage}\label{wxlatexdcstartpage}

\func{bool}{StartPage}{\void}

Does nothing
